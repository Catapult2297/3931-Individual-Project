\documentclass[11pt,a4paper]{article}
\usepackage[T1]{fontenc}
\usepackage{tgbonum}
\usepackage{hyperref}

%Stop auto indent
\usepackage{parskip}

%Enhancements for improving the information structure and printed output
\usepackage{amsmath}

%Provides an extended symbol collection
\usepackage{amssymb}

%Helps to define theorem-like structures
\usepackage{amsthm}

%Doublestroke package
%\usepackage{dsfont}

%Load pictures into document
\usepackage{graphicx}
\graphicspath{ {./images/} }

%Fancy headers
\usepackage{fancyhdr}

%Page size and margins
\usepackage[a4paper,left=3cm,right=1.5cm,top=2cm,bottom=2cm]{geometry}

%An extended implementation of the array and tabular environments
\usepackage{array}
\newcolumntype{L}{>{$}c<{$}} %Use ``L'' in tabular to write maths

\usepackage{esint}
%Program code related packages
%Use \begin{lstlisting} and \end{lstlisting} to create program code
%environment
\usepackage{listings}
\usepackage[svgnames]{xcolor}
\usepackage{fontspec}
\setmonofont[Scale=0.85]{FiraCode Nerd Font}
\definecolor{backcolour}{rgb}{0.95,0.95,0.92}
\lstset{
  frame=tb,
  backgroundcolor=\color{backcolour},
  language=Python, %Change this for other programming languages e.g. Python, Java
  aboveskip=3mm,
  belowskip=3mm,
  showstringspaces=false,
  columns=flexible,
  basicstyle={\footnotesize\ttfamily},
  numbers=left,
  numberstyle=\tiny\color{lightgray},
  keywordstyle=\color{red},
  commentstyle=\color{Green},
  stringstyle=\color{blue},
  breaklines=true,
  breakatwhitespace=false,
  tabsize=4
}
\begin{document}

\pagestyle{fancy}
%Module Name
\fancyhead[L]{\footnotesize{COMP3931 Individual Project}}
%Document Name
\fancyhead[R]{\footnotesize{Project Outline}}
%Name and Student ID
\fancyfoot[R]{\footnotesize Wing Fung Choy [201519376]}

\newcommand{\vect}[1]{\underline{\boldsymbol{#1}}}
\newcommand{\headingtitle}[1]{\textbf{#1}\normalsize\\}
\newcommand{\keyword}[1]{\colorbox{lightgray}{\texttt{#1}}}
%===============================
\begin{center}
    \textbf{COMP3931}\\
    Individual Project\\
\end{center}
\vspace{11pt}
\begin{center}
    Project Outline
    \vspace{11pt}
\end{center}

Summary:\\
This project aims to develop a Basic Rust Code Logic Analyser specifically 
designed for the Rust programming language. Drawing inspiration from foundational 
works in the field, including the 1973 paper "Automatic Program Verification I: 
A Logical Basis and Its Implementation" and the 1976 paper "Automatic Program 
Verification V: Verification-Oriented Proof Rules for Arrays, Records, and Pointers".
The project aims to integrate classical verification techniques with modern 
programming paradigms.

By leveraging Hoare logic, the verifier establishes a formal 
framework for reasoning about program correctness. The implementation seeks to 
adapt these established principles to the unique features and safety guarantees 
provided by Rust, thereby enhancing the reliability of software development in a 
contemporary context.

This analyser aims to provide developers with automated tools to verify the correctness
of their Rust programs, ensuring adherence to specified properties and reducing 
the likelihood of errors.

%\vspace{22pt}
%Student: \author1{Wing Fung Choy - \href{mailto:sc21wfc@leeds.ac.uk}{sc21wfc@leeds.ac.uk}}

%\vspace{11pt}
%Supervisor: Professor Dillon Mayhew - \href{mailto:D.Mayhew@leeds.ac.uk}{D.Mayhew@leeds.ac.uk}
%\vspace{11pt}

\pagebreak
\section{Project Scope}
The primary focus of this analyser is to evaluate basic data types and control flow
within Rust programs. By concentrating on these fundamental aspects, the analyser aims
to provide effective verification without the complexities associated with fully 
verifying extensive codebases. This targeted approach allows for a more manageable 
analysis process, ensuring that essential properties of the code can be validated 
efficiently.

It is important to note that the analyser will not encompass a full termination 
analysis of programs. Given that the halting problem is undecidable, attempting to 
ascertain whether arbitrary programs will terminate is outside the scope of this 
project. Instead, the focus will remain on verifying the correctness of operations 
within the defined parameters of basic data types and control structures.

Additionally, this analyser will be limited in the types of statements it can 
interpret, adhering closely to the constraints outlined in the foundational papers 
on automatic program verification. By imposing these limitations, the analyser can 
maintain a clear and focused methodology, ensuring that the verification process 
remains both feasible and effective. This approach allows for meaningful insights 
into the behaviour of the code while recognising the inherent challenges in broader 
program verification.

\pagebreak
\section{Challenges and Difficulities}
These are the potential challenges that may arise during development:
\begin{enumerate}
    \item Rust has a sophisticated type system that includes features like ownership,
          borrowing, and lifetimes. Accurately modelling these concepts in an analyser can be 
          complex, as they introduce nuances that traditional analysers may not 
          account for.
    \item Integrating the analyser with existing Rust tooling and workflows can
          present challenges. Ensuring compatibility and usability within the Rust 
          ecosystem will require careful design and implementation.
    \item As Rust continues to evolve, keeping the analyser updated with the latest
          language features and paradigms will require ongoing effort and adaptability.
    \item Achieving both soundness and completeness in the analyzer is a critical
          challenge.
\end{enumerate}

\pagebreak
\section{Implementation Plan}
This project will split into these key steps:
\begin{enumerate}
    \item Understand the original papers, Hoare logic and the Pascal Language.
    \item Create a program that will take a Rust program (or snippet) as an input,
          and parse the program to generate an Abstract Syntax Tree.
    \item Generate or take a user's input as the proof goal for the analyser.
    \item Generate a complete proof of the program using techniques such as
          Natural Deduction or Resolution
\end{enumerate}

\pagebreak
\section{Ethical Problems}
Here are some ethical problems that will need to be considered during development:
\begin{itemize}
    \item Software Reliability and Safety.
    \item User Trust and Transparency.
    \item Intellectual Property.
    \item Bias in Coding Styles.
    \item Job Displacement.
    \item Security Vulnerability.
    \item Responsibility for Errors.
    \item Accessibility.
    \item Computational Resources.
\end{itemize}
%===============================
\end{document}
